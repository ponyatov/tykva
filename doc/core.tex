\clearpage\secrel{Базовый язык}\label{core}\secdown

Задача реализации \emph{базового языка}\ --- максимально быстро и просто
получить работающий интерпретатор. На этом этапе не идёт разговор об
эффективности, фичах и т.д. Для начала нужно сделать MVP, и хотя бы убедиться
что бредовая идея программирования в кириллице вообще принципиально
жизнеспособна.

Первой проблемой будет запрет переключения клавиатуры на латинскую раскладку,
для ввода привычных операторов\ -- если уж вам так печот от англицизмов в языке
программирования, то и с этим справитесь. Остаются доступными только операторы,
доступные в типовой кириллической раскладке:
\begin{itemize}[nosep]
    \item арифметические операторы: \verb$+$ \verb$-$ \verb$*$ \verb$\$ \verb$%$
    \item знак равенства \verb$=$ при том что символы сравнения \verb$<$ \verb$>$ недоступны
    \item знаки препинания \verb$!$ \verb$;$ \verb$%$ \verb$:$ \verb$?$
    \item скобки \verb$($ \verb$)$
\end{itemize}

\secrel{Комментарии}

\begin{verbatim}
№ строчный комментарий
\end{verbatim}

взят символ, находящийся на той же кнопке что и классический \#

\secup
